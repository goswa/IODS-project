\documentclass[]{article}
\usepackage{lmodern}
\usepackage{amssymb,amsmath}
\usepackage{ifxetex,ifluatex}
\usepackage{fixltx2e} % provides \textsubscript
\ifnum 0\ifxetex 1\fi\ifluatex 1\fi=0 % if pdftex
  \usepackage[T1]{fontenc}
  \usepackage[utf8]{inputenc}
\else % if luatex or xelatex
  \ifxetex
    \usepackage{mathspec}
  \else
    \usepackage{fontspec}
  \fi
  \defaultfontfeatures{Ligatures=TeX,Scale=MatchLowercase}
\fi
% use upquote if available, for straight quotes in verbatim environments
\IfFileExists{upquote.sty}{\usepackage{upquote}}{}
% use microtype if available
\IfFileExists{microtype.sty}{%
\usepackage{microtype}
\UseMicrotypeSet[protrusion]{basicmath} % disable protrusion for tt fonts
}{}
\usepackage[margin=1in]{geometry}
\usepackage{hyperref}
\hypersetup{unicode=true,
            pdftitle={Summarizing learning of this week},
            pdfauthor={Shweta Goswami},
            pdfborder={0 0 0},
            breaklinks=true}
\urlstyle{same}  % don't use monospace font for urls
\usepackage{graphicx,grffile}
\makeatletter
\def\maxwidth{\ifdim\Gin@nat@width>\linewidth\linewidth\else\Gin@nat@width\fi}
\def\maxheight{\ifdim\Gin@nat@height>\textheight\textheight\else\Gin@nat@height\fi}
\makeatother
% Scale images if necessary, so that they will not overflow the page
% margins by default, and it is still possible to overwrite the defaults
% using explicit options in \includegraphics[width, height, ...]{}
\setkeys{Gin}{width=\maxwidth,height=\maxheight,keepaspectratio}
\IfFileExists{parskip.sty}{%
\usepackage{parskip}
}{% else
\setlength{\parindent}{0pt}
\setlength{\parskip}{6pt plus 2pt minus 1pt}
}
\setlength{\emergencystretch}{3em}  % prevent overfull lines
\providecommand{\tightlist}{%
  \setlength{\itemsep}{0pt}\setlength{\parskip}{0pt}}
\setcounter{secnumdepth}{0}
% Redefines (sub)paragraphs to behave more like sections
\ifx\paragraph\undefined\else
\let\oldparagraph\paragraph
\renewcommand{\paragraph}[1]{\oldparagraph{#1}\mbox{}}
\fi
\ifx\subparagraph\undefined\else
\let\oldsubparagraph\subparagraph
\renewcommand{\subparagraph}[1]{\oldsubparagraph{#1}\mbox{}}
\fi

%%% Use protect on footnotes to avoid problems with footnotes in titles
\let\rmarkdownfootnote\footnote%
\def\footnote{\protect\rmarkdownfootnote}

%%% Change title format to be more compact
\usepackage{titling}

% Create subtitle command for use in maketitle
\newcommand{\subtitle}[1]{
  \posttitle{
    \begin{center}\large#1\end{center}
    }
}

\setlength{\droptitle}{-2em}

  \title{Summarizing learning of this week}
    \pretitle{\vspace{\droptitle}\centering\huge}
  \posttitle{\par}
    \author{Shweta Goswami}
    \preauthor{\centering\large\emph}
  \postauthor{\par}
      \predate{\centering\large\emph}
  \postdate{\par}
    \date{11-11-2018}


\begin{document}
\maketitle

new\_data \textless{}-
read.table(``\url{http://s3.amazonaws.com/assets.datacamp.com/production/course_2218/datasets/learning2014.txt}'',
sep=``,'', header=TRUE)

\begin{itemize}
\tightlist
\item
  The current dataset is a part of the course `Introduction to Social
  Statistics' held by `Kimmo Vehkalahti' in `Fall 2014'.The study data
  were from the years 2014 to 2015 and it aimed to summarize the study
  behavior and approaches to learning in an international survey.
\end{itemize}

str(new\_data) dim(new\_data) head(new\_data, n = 20)

\begin{itemize}
\tightlist
\item
  Out of the 183 observations of 60 variables, the current dataset has
  166 observations of 7 variables. The selected variables for analysis
  from the raw dataset were gender, age, attitude, deep questions,
  strategic questions, surface questions and exam points.
\end{itemize}

\subsubsection{Graphical overview of the data and summaries of the
variables in the
data.}\label{graphical-overview-of-the-data-and-summaries-of-the-variables-in-the-data.}

library(GGally) library(ggplot2) p \textless{}- ggpairs(new\_data,
mapping = aes(col = gender, alpha = 0.3), lower = list(combo =
wrap(``facethist'', bins = 20))) print(p)

\begin{itemize}
\tightlist
\item
  The corelation matrix plot was created using function ggpairs. The
  corelation between age, attitude, deep, stra, surf with exam points
  were -0.093, ``0.437'', -0.01, 0.146, -0.144 resp. The highest
  corelation was found between attitude and exam points.
\end{itemize}

\subsubsection{Describe and interpret the outputs, commenting on the
distributions of the variables and the relationships between
them.}\label{describe-and-interpret-the-outputs-commenting-on-the-distributions-of-the-variables-and-the-relationships-between-them.}

library(Hmisc) describe(new\_data)

summary(new\_data)

\begin{itemize}
\item
  The male and female participants were 56 and 110 resp. The minumum and
  maximum age of the participants were 17 and 55 resp. The mean age was
  25.51.
\item
  The attitude variable had minimum and maximum values 1.40 and 5.00
  resp. Considering 5-point scale (Likert Scale), the mean attitude
  showed neutral response with value 3.14.
\item
  The deep, strategic and surface learning had mean values 3.68, 3.12,
  2.78 resp. The deep learning had slightly high mean than the strategic
  learning. Therefore, the deep learning was the prevailing learning
  approach as compared to strategic and surface ones.
\item
  Regarding exam points, the maximum and minimum points were 33 and 7
  resp. The mean points were 22.7.
\end{itemize}

\subsubsection{Choose three variables as explanatory variables and fit a
regression model where exam points is the target (dependent) variable.
Show a summary of the fitted model and comment and interpret the
results.}\label{choose-three-variables-as-explanatory-variables-and-fit-a-regression-model-where-exam-points-is-the-target-dependent-variable.-show-a-summary-of-the-fitted-model-and-comment-and-interpret-the-results.}

my\_model \textless{}- lm(points \textasciitilde{} attitude + surf +
stra, data = new\_data) summary(my\_model)
summary(my\_model)\$coefficients

\begin{itemize}
\tightlist
\item
  The three variables selected for regression analysis were attitude,
  surface learning and strategic learning. The regression output showed
  that the strategic and surface learning were not statistically
  significant. On the other hand, attitude seemed to be statistically
  significant as its value is less than the p-significance level (0.05).
  Taken into consideration the coefficient p-values and to increase the
  model precision, strategic and surface learning will be excluded in
  the final model.
\end{itemize}

\subsubsection{If an explanatory variable in your model does not have a
statistically significant relationship with the target variable, remove
the variable from the model and fit the model again without
it}\label{if-an-explanatory-variable-in-your-model-does-not-have-a-statistically-significant-relationship-with-the-target-variable-remove-the-variable-from-the-model-and-fit-the-model-again-without-it}

my\_model2 \textless{}- lm(points \textasciitilde{} attitude, data =
new\_data) summary(my\_model2)

\subsubsection{Using a summary of your fitted model, explain the
relationship between the chosen explanatory variables and the target
variable (interpret the model parameters). Explain and interpret the
multiple R squared of the
model.}\label{using-a-summary-of-your-fitted-model-explain-the-relationship-between-the-chosen-explanatory-variables-and-the-target-variable-interpret-the-model-parameters.-explain-and-interpret-the-multiple-r-squared-of-the-model.}

\begin{itemize}
\item
  The residuals were approximately normally distributed. The higher the
  t-value, the better it is.The model showed t-value (6.21) which should
  be more than 2, thus, fairly reliable coefficient as a predictor.
\item
  The relation between attitude and exam points existed as p-value
  (4.12e-09 ***) was less than the pre-determined statistical
  significance level (0.05).
\item
  The standard error should be less than 2.5\% to have the required
  precsion. In our model, the Standard Error is 0.6\%.
\item
  The R-square value in the model was 0.19. The variation of dependent
  variable (exam points) with the independent one (attitude) was 19\%
  and is not quite high. The regression as a whole fits the data quite
  well.
\end{itemize}

\subsubsection{Produce the following diagnostic plots: Residuals vs
Fitted values, Normal QQ-plot and Residuals vs
Leverage.}\label{produce-the-following-diagnostic-plots-residuals-vs-fitted-values-normal-qq-plot-and-residuals-vs-leverage.}

my\_model \textless{}- lm(points \textasciitilde{} attitude + surf +
stra, data = new\_data) par(mfrow=c(2,2)) plot(my\_model, which =
c(1,2,5))

\subsubsection{Explain the assumptions of the model and interpret the
validity of those assumptions based on the diagnostic
plots.}\label{explain-the-assumptions-of-the-model-and-interpret-the-validity-of-those-assumptions-based-on-the-diagnostic-plots.}

\begin{itemize}
\item
  Residual vs Fitted plot:

\begin{verbatim}
    + This plot is used to determine non-linearity. The plot seemed OK.The data appeared to be randomly spread around the straight line except for some points (35,56 and 145). The straight line showed no non-linear trend to the residuals.
\end{verbatim}
\item
  Normal Q-Q

\begin{verbatim}
    + This plot is used to determine whether the residuals are normally distributed or not. The residuals in this plot were slightly deviated from the diagonal line in both the upper and lower tail. Points 35, 145 and 56 looked a little off here. The residuals were approximately normally distributed.
\end{verbatim}
\item
  Residuals vs Leverage

\begin{verbatim}
    + This plot is used to find influential cases. The plot seemed to have no influential cases. The Cook's distance line/dashed curves did not appear on the plot.
\end{verbatim}
\end{itemize}

\subsubsection{References}\label{references}

\begin{itemize}
\item
  \url{https://campus.datacamp.com/courses/helsinki-open-data-science/regression-and-model-validation?ex=1}
\item
  \url{http://www.helsinki.fi/~kvehkala/JYTmooc/JYTOPKYS3-meta.txt}
\item
  \url{http://statisticsbyjim.com/regression/standard-error-regression-vs-r-squared/}
\end{itemize}


\end{document}
